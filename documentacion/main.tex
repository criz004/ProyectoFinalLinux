\documentclass{article}

\usepackage[utf8]{inputenc}
\usepackage{graphicx} % Required for inserting images
\usepackage{listings}
\usepackage{xcolor}
\usepackage[spanish]{babel}
\usepackage{float}

\title{Proyecto Final Linux Terminal}
\author{Frida Hernández Rodríguez, Cristopher Velasco Ávila}
\date{Abril 2025}

\begin{document}

\maketitle
\pagebreak
\tableofcontents
\pagebreak
\section{Introducción}
El presente proyecto consiste en el desarrollo de un programa que simula una terminal de trabajo en un sistema operativo basado en Linux.  
El objetivo principal es crear un entorno personalizado donde el usuario pueda interactuar mediante comandos específicos programados en scripts de Bash.  
Entre las funciones disponibles se encuentran la visualización de información del sistema, consultar fecha y hora actual, búsqueda de archivos, comando de ayuda para conocer los comandos disponibles, reproducción de música, ejecución de un juego, entre otros.  

Entre las particularidades del proyecto se encuentra un sistema de autenticacion para acceso a usuarios existentes en el SO anfitrión. El cual hace que al ejecutar el script principal del proyecto, lo primero por hacer es introducir un usuario y contraseña para poder ingresar. De igual manera, se cuenta con una linea de comandos que muestra el usuario activo, el nombre del host y la carpeta donde se encuentra. La terminal es capaz de ejecutar los comandos programados por los miembros de este proyecto asi como los disponibles en el SO anfitrión, a excepción del comando -exit- el cual fue deshabilitado, al igual que Ctrl+C y Ctrl+Z, con el objetivo de que solo se pueda salir de la terminal utilizando el comando -salir- programado por los integrantes del proyecto.

Cada comando fue desarrollado mediante scripts independientes que son invocados desde un programa principal que gestiona el flujo de la terminal personalizada. Los comandos programados que ejecuta este terminal son los siguientes:

\begin{itemize}
    \item \textbf{proyecto}: Este es el encargado de iniciar el proceso de la terminal y comienza con el login.
    \item \textbf{ayuda}: Comando que muestra una lista de los comandos disponibles y una breve descripcion de que es lo que hacen.
    \item \textbf{creditos}: Comando que muestra los nombres de los integrantes del proyecto.
    \item \textbf{buscar}: Este comando busca un archivo en una carpeta.
    \item \textbf{infosis}: Comando que muestra la información del sistema:
    \item \textbf{hora}: Comando que muestra la fecha y hora actual.
    \item \textbf{juego}: Comando que inicia el juego gato.
    \item \textbf{musica}: Comando que arroja el reproductor mp3.
    \item \textbf{salir}: Comando que termina el proceso de -proyecto- es decir, la terminal personalizada. 
\end{itemize}

\pagebreak
  \section{Desarrollo}
  \subsection{Script Principal: proyecto}

El script llamado -proyecto- constituye el script principal del proyecto, actuando como una terminal personalizada. Permite al usuario autenticarse en el sistema, proporciona un menú de comandos personalizados y controla la interacción con los demás scripts.
Este script se ejecuta cuando el usuario ingresa \texttt{./proyecto.sh} en la terminal.

\subsubsection{Captura de pantalla de la ejecución}

\begin{figure}[!ht]
\centering
\includegraphics[width=0.7\textwidth]{Login.png}
\caption{Ejecución del script proyecto - LOGIN}
\end{figure}

\begin{figure}[h!]
\centering
\includegraphics[width=0.7\textwidth]{bienvenida.png}
\caption{Ejecución del script de proyecto - Linea de comandos}
\end{figure}

\subsubsection{Código fuente del script}

\begin{lstlisting}[language=bash]
# !/bin/bash
# Proyecto de bash

trap '' SIGINT
trap '' SIGTSTP

# LOGIN
tput clear
echo "Hola hola, bienvenido a esta terminal"
#Usuario
read -p "Usuario: " usuario

if id "$usuario" &>/dev/null; then
    echo "Usuario encontrado"
else 
    echo "No se encontro el usuario"
    exit 1
fi

#Contrasena
read -s -p Contrasena:
echo

if [ -z "$contrasena" ]; then
    echo "No ingresaste ninguna contrasena. Intenta de nuevo."
    exit 1
fi

echo "$contrasna" | sudo -S -u "$usuario" whoami &>/dev/null

if [ $? -eq 0 ] ; then
   echo "Todo cool, puedes entrar $usuario"
else 
  echo " Ey te equivocaste en la contrasena, no puedes entrar, intenta de nuevo"
  exit 1
fi

sleep 2
clear

command="null"
guia=$(pwd)
cd ~

export PROYECTO_PID=$$

echo "Proyecto Bash en Linux"
echo "Version 1.0.0"
echo "Para ayuda de comandos, escriba "ayuda""
echo
sleep 2

while true;
do
    read -p "${usuario}@${HOSTNAME} $(pwd) -> " command
    
    case $command in

    ayuda)
        echo
        $guia/comandos/ayudaa.sh
        echo
        ;;

    creditos)
        $guia/comandos/creditos.sh
        ;;

    buscar)
        $guia/encuentra.sh
        ;;
    
    infosis)
        $guia/infosis.sh
        ;;
    
    hora)
        $guia/hora.sh
        ;;
    
    juego)
        $guia/gato.sh
        ;;

    musica)
        echo "Aqui va el reproductor.sh"
        ;;

    salir)
        $guia/comandos/salir.sh
        ;;

    exit)
        echo "Comando deshabilitado. Usa el comando -salir- para cerrar
        la terminal"
        ;;
    *)
        $command
        ;;
    esac
done

\end{lstlisting}

  \subsection{Script: ayuda}

El script de ayuda proporciona al usuario una lista de los comandos disponibles dentro de la terminal simulada, junto con una breve descripción de cada uno.  
Este script se ejecuta cuando el usuario ingresa el comando \texttt{ayuda} en la terminal.

\subsubsection{Captura de pantalla de la ejecución}

\begin{figure}[h!]
\centering
\includegraphics[width=0.7\textwidth]{ayuda.png}
\caption{Ejecución del script de ayuda}
\end{figure}

\subsubsection{Código fuente del script}

\begin{lstlisting}[language=bash]
#!/bin/bash

# Script de ayuda para mostrar comandos disponibles

echo "=========== <3 AYUDITA PARA CONOCER COMANDOS DISPONIBLES <3 ==========="
echo ""
echo "1. ayuda     - Muestra esta lista de comandos"
echo "2. creditos  - Muestra los integrantes del proyecto"
echo "3. buscar    - Busca un archivo en una carpeta"
echo "4. infosis   - Muestra la informacion del sistema"
echo "5. hora      - Muestra la hora y fecha actual"
echo "6. juego     - Inicia el juego gato# "
echo "7. musica    - Arroja el reproductor MP3"
echo "8. salir     - Cierra la terminal"
echo ""
echo "========================================================================"

\end{lstlisting}
\pagebreak
\subsection{Script: creditos}

El script de creditos tiene como propósito mostrar de manera visual y estilizada los créditos del proyecto, destacando a los desarrolladores de forma artística mediante figuras de texto ASCII.
El script no recibe parámetros de entrada ni requiere interacción del usuario. Su única función es desplegar la información visualmente atractiva en la terminal.  
Este script se ejecuta cuando el usuario ingresa el comando \texttt{creditos} en la terminal.

\subsubsection{Captura de pantalla de la ejecución}

\begin{figure}[h!]
\centering
\includegraphics[width=0.7\textwidth]{creditos.png}
\caption{Ejecución del script de creditos}
\end{figure}

\subsubsection{Código fuente del script}

\begin{lstlisting}[language=bash]
#!/bin/bash

echo "=================================================================================================================="
echo "   ___  ____  ____  ____  __  ____  __   ____    ____  ____  __      ____  ____   __  _  _  ____  ___  ____  __  "
echo "  / __)(  _ \(  __)(    \(  )(_  _)/  \ / ___)  (    \(  __)(  )    (  _ \(  _ \ /  \( \/ )(  __)/ __)(_  _)/  \ " 
echo " ( (__  )   / ) _)  ) D ( )(   )( (  O )\___ \   ) D ( ) _) / (_/\   ) __/ )   /(  O ))  /  ) _)( (__   )( (  O ) "
echo "  \___)(__\_)(____)(____/(__) (__) \__/ (____/  (____/(____)\____/  (__)  (__\_) \__/(__/  (____)\___) (__) \__/  "
echo "=================================================================================================================="
echo ""
echo "DESARROLLADO POR:"
echo ""
echo "FRIDA HERNANDEZ"
echo "Y"
echo "CRISTOPHER VELASCO"
               
\end{lstlisting}

\pagebreak
\subsection{Script: buscar}

El script de buscar permite al usuario verificar si un archivo existe dentro de un directorio específico.
Si el archivo no existe, ofrece la opción de crearlo automáticamente en la ruta indicada.
Su propósito es facilitar la gestión básica de archivos mediante una interfaz sencilla de línea de comandos, mejorando la eficiencia y la interacción del usuario con el sistema de archivos. 
Este script se ejecuta cuando el usuario ingresa el comando \texttt{buscar} en la terminal.

\subsubsection{Captura de pantalla de la ejecución}

\begin{figure}[h!]
\centering
\includegraphics[width=0.7\textwidth]{buscar2.png}
\caption{Ejecución del script de buscar}
\end{figure}

\subsubsection{Código fuente del script}

\begin{lstlisting}[language=bash]
#!/bin/bash

#Leemos el directorio
read -p "Ingresa el nombre del directorio: " path

#Ingresa diagonal al final de path si no la tiene
if [[ ! ${path} == */ ]]; then
    path="${path}/"
fi

#Revisamos si el directorio existe
if [[ ! -d ${path} ]]; then
    echo " Error! El directorio no existe. Abortando..."
    exit
fi

#Leemos el nombre del archivo
read -p "Ingresa el nombre del archivo: " name

#Revisamos si el archivo existe como tal
if [ -e "${path}${name}" ]; then
    echo "El archivo si existe"
else 
    read -p "El archivo no existe. Deseas crearlo? (s/*): " opcion
    if [[ $opcion == "s" || $opcion == "S" ]]; then
        touch ${path}${name}
    fi
fi 
                                                 
\end{lstlisting}
\pagebreak
\subsection{Script: infosis}

El script de infosis muestra de manera sencilla y rápida tres tipos de información sobre el sistema operativo:

\begin{itemize}
    \item Cantidad de Memoria RAM disponible
    \item Arquitectura del procesador (si es de 32 o 64 bits)
    \item Versión y detalles del sistema operativo
\end{itemize}

Es muy útil para hacer diagnósticos básicos o simplemente para conocer las características de la computadora donde se está trabajando.
Este script se ejecuta cuando el usuario ingresa el comando \texttt{infosis} en la terminal.

\subsubsection{Captura de pantalla de la ejecución}

\begin{figure}[h!]
\centering
\includegraphics[width=0.7\textwidth]{infosis.png}
\caption{Ejecución del script de infosis}
\end{figure}

\subsubsection{Código fuente del script}

\begin{lstlisting}[language=bash]
#!/bin/bash

echo "=== Memoria RAM ==="
awk '/MemTotal/ {print "Total RAM: " $2/1024 " MB"}' /proc/meminfo

echo -e "\n=== Architectura ==="
cat /proc/cpuinfo | grep flags | grep -q ' lm ' && echo "64-bit" ||
    echo "32-bit"

echo -e "\n=== Version del sistema ==="
cat /etc/os-release
                                                 
\end{lstlisting}
\pagebreak
\subsection{Script: hora}

El script de hora lee la hora y la fecha actuales desde el archivo /proc/driver/rtc, que contiene información sobre la hora del sistema. Luego, ajusta la hora en función de una zona horaria específica, en este caso, la zona horaria con un offset de -6 horas. El script maneja la conversión de hora y fecha, ajustando tanto la hora como el día si es necesario (por ejemplo, si la hora ajustada es menor a 0 o mayor a 23). Finalmente, muestra la hora y la fecha ajustadas en formato HH:MM:SS y AAAA:MM:DD.  El objetivo de este script es poder ajustar la hora y fecha del sistema a una zona horaria diferente a la local. 
Este script se ejecuta cuando el usuario ingresa el comando \texttt{hora} en la terminal.

\subsubsection{Captura de pantalla de la ejecución}

\begin{figure}[h!]
\centering
\includegraphics[width=0.7\textwidth]{hora.png}
\caption{Ejecución del script de hora}
\end{figure}

\subsubsection{Código fuente del script}

\begin{lstlisting}[language=bash]
#!/bin/bash

# Leemos hora y fecha desde el archivo /proc/driver/rtc
hora=$(grep -o -E -m 1 '[0-9]{2}:[0-9]{2}:[0-9]{2}' /proc/driver/rtc)
fecha=$(grep -o -E -m 1 '[0-9]{4}-[0-9]{2}-[0-9]{2}' /proc/driver/rtc)

# Separamos la hora en sus tres componentes
horas=${hora:0:2}
minutos=${hora:3:2}
segundos=${hora:6:2}

# Separamos la fecha en tres componentes
anio=${fecha:0:4}
meses=${fecha:5:2}
dias=${fecha:8:2}

# Zona horaria nueva
offset=-6

hora_nueva=$((10#$horas + offset))

# Ajustar fecha si es necesario

if [ "$hora_nueva" -lt 0 ]; then
    hora_nueva=$((hora_nueva + 24))
    dias=$((10#$dias - 1))
fi

if [ "$hora_nueva" -ge 24 ]; then
    hora_nueva=$((hora_nueva - 24))
    dias=$((10#$dias + 1))
fi

echo "Hora: ${hora_nueva}:${minutos}:${segundos}"
echo "Fecha: ${anio}-${meses}-${dias}"

\end{lstlisting}
\pagebreak
\subsection{Script: juego}

El script de juego implementa el clásico juego de Gato para dos jugadores en la terminal. Usa un arreglo para representar el tablero y alterna turnos entre las marcas X, O. Cada jugador elige una posición disponible, validándose que sea un movimiento válido. Tras cada turno, se verifica si hay un ganador o si se produce un empate. El tablero se actualiza y muestra en cada jugada, permitiendo seguir el progreso del juego. 
Este script se ejecuta cuando el usuario ingresa el comando \texttt{juego} en la terminal.

\subsubsection{Captura de pantalla de la ejecución}

\begin{figure}[h!]
\centering
\includegraphics[width=0.7\textwidth]{juego.png}
\caption{Ejecución del script de juego}
\end{figure}

\subsubsection{Código fuente del script}

\begin{lstlisting}[language=bash]
#!/bin/bash

echo "Bienvenido a Gato"

tablero=("#" "#" "#" "#" "#" "#" "#" "#" "#")

function imprimir_tablero() {
    echo "Tablero actual:"
    echo "${tablero[0]} | ${tablero[1]} | ${tablero[2]}"
    echo "---------"
    echo "${tablero[3]} | ${tablero[4]} | ${tablero[5]}"
    echo "---------"
    echo "${tablero[6]} | ${tablero[7]} | ${tablero[8]}"
}

function revisar_ganador(){

    # Check de horizontales
    if [[ ${tablero[0]} == ${tablero[1]} && 
      ${tablero[1]} == ${tablero[2]} && ${tablero[0]} != "#" ]]; then
        ganador="${tablero[0]}"
    else if [[ ${tablero[3]} == ${tablero[4]} && ${tablero[4]} == ${tablero[5]}
      && ${tablero[3]} != "#" ]]; then
        ganador="${tablero[3]}"
    else if [[ ${tablero[6]} == ${tablero[7]} && ${tablero[7]} == ${tablero[8]} 
      && ${tablero[6]} != "#" ]]; then
        ganador="${tablero[6]}"
    # Check de verticales
    else if [[ ${tablero[0]} == ${tablero[3]} && ${tablero[3]} == ${tablero[6]} 
      && ${tablero[0]} != "#" ]]; then
        ganador="${tablero[0]}"
    else if [[ ${tablero[1]} == ${tablero[4]} && ${tablero[4]} == ${tablero[7]} 
      && ${tablero[1]} != "#" ]]; then
        ganador="${tablero[1]}"
    else if [[ ${tablero[2]} == ${tablero[5]} && ${tablero[5]} == ${tablero[8]} 
      && ${tablero[2]} != "#" ]]; then
        ganador="${tablero[2]}"
    # Check de diagonales
    else if [[ ${tablero[0]} == ${tablero[4]} && ${tablero[4]} == ${tablero[8]} 
      && ${tablero[0]} != "#" ]]; then
        ganador="${tablero[0]}"
    else if [[ ${tablero[2]} == ${tablero[4]} && ${tablero[4]} == ${tablero[6]} 
      && ${tablero[2]} != "#" ]]; then
        ganador="${tablero[2]}"
    else if [[ ${numeroTurno} == 10 ]]; then
        ganador="Empate"
    fi fi fi fi fi fi fi fi fi
}

ganador=0
turnoActual=X
numeroTurno=1

while [ $ganador == 0 ] 
do

    if [[ ${turnoActual} == "X" && ${error} == 0 ]]; then
        turnoActual="O"
    else if [[ ${turnoActual} == "O" && ${error} == 0 ]]; then
        turnoActual="X"
    fi fi

    echo "-----------------------------------"
    echo "Turno numero $numeroTurno! toca a ${turnoActual}"
    imprimir_tablero
    
    error=0

    read -p "Elige posicion (1-9): " posicion

    if [[ posicion -gt 9 || posicion -lt 1 ]]; then
        echo "Posicion no valida, elige otra!"
        error=1
    else if [ ! ${tablero[$posicion-1]} == "#" ]; then
        echo "Posicion ocupada, elige otra!"
        error=1
    else if [ ${tablero[$posicion-1]} == "#" ]; then
        tablero[$posicion-1]="$turnoActual"
    fi fi fi

    if [  $error == 0 ]; then
        numeroTurno=$(( numeroTurno + 1 ))
    fi

    revisar_ganador



done

echo "-----------------------------------"
imprimir_tablero

if [[ ${ganador} == "Empate" ]]; then
    echo "Empate! Nadie gana"
    echo
else
    echo "El ganador es $ganador!"
    echo
fi
exit

\end{lstlisting}
\pagebreak
\subsection{Script: musica}

El script de musica actúa como un reproductor de música en la terminal, permitiendo al usuario seleccionar y reproducir canciones MP3 desde un directorio específico. Pide al usuario el nombre del directorio, verifica su existencia, y muestra una lista de canciones. Luego, permite reproducir canciones con comandos para avanzar, retroceder o salir, y continúa hasta que el usuario decida salir. 
Este script se ejecuta cuando el usuario ingresa el comando \texttt{musica} en la terminal.

\subsubsection{Captura de pantalla de la ejecución}

\begin{figure}[h!]
\centering
\includegraphics[width=0.7\textwidth]{mp31.jpg}
\caption{Ejecución del script de musica}
\end{figure}

\begin{figure}[h!]
\centering
\includegraphics[width=0.7\textwidth]{mp32.jpg}
\caption{Ejecución del script de musica}
\end{figure}

\begin{figure}[h!]
\centering
\includegraphics[width=0.7\textwidth]{mp33.jpg}
\caption{Ejecución del script de musica}
\end{figure}
\pagebreak
\subsubsection{Código fuente del script}

\begin{lstlisting}[language=bash]
#!/bin/bash
#Pregunta nombre de carpeta
read -p "Ingresa el nombre del directorio de musica (debe estar ubicado en 
/home/${USERNAME}!): " carpeta

#Guarda la carpeta en MUSIC_DIR
MUSIC_DIR="/home/${USERNAME}/${carpeta}"
clear

if [ ! -d "$MUSIC_DIR" ]; then
    echo "Error! no existe tal directorio!"
    exit 1
fi

# Guardamos MP3 a un arreglo
CANCIONES=("$MUSIC_DIR"/*.mp3)
NO_CANCIONES=${#CANCIONES[@]}

# Mostramos la lista de canciones encontradas
echo "=== REPRODUCTOR MP3 ==="
echo "CANCIONES ENCONTRADAS :D"
for i in "${!CANCIONES[@]}"; do #Buscamos por indices, no por el nombre
    echo " [$i] $(basename "${CANCIONES[$i]}")"
done

echo
read -p "Ingresa el indice de la cancion: " INDICE_INI

#Verificamos si el indice esta dentro del intervalo valido
if [[ "$INDICE_INI" -lt 0 || "$INDICE_INI" -ge "$NO_CANCIONES" ]]; then
    echo "Invalid indice."
    exit 1
fi

#Imprimimos controles
echo "Controles: [K] - Anterior, [L] - Siguiente, [Q] - Salir"


indice=$INDICE_INI

while [ $indice -ge 0 ] && [ $indice -lt $NO_CANCIONES ]; do
    CANCION_ACTUAL=$(basename "${CANCIONES[$indice]}")
    echo -e "\nPlaying: $CANCION_ACTUAL"
    
    #Reproducir cancion
    mpg123 -q "${CANCIONES[$indice]}" &
    PID=$!
    actual=""

    #Verificamos si el usuario escribe algo en teclado
    while kill -0 $PID 2>/dev/null; do
        read -rsn1 -t 0.1 key
        if [[ $key == "l" ]]; then
            actual="siguiente"
            break
        elif [[ $key == "k" ]]; then
            actual="anterior"
            break
        elif [[ $key == "q" ]]; then
            kill $PID
            wait $PID 2>/dev/null
            echo "Saliendo. Adios!"
            exit 0
        fi
    done

    #Se realiza accion segun la entrada de usuario
    if [[ $actual == "siguiente" ]]; then
        kill $PID
        wait $PID 2>/dev/null
        indice=$((indice + 1))
        
    elif [[ $actual == "anterior" ]]; then
        kill $PID
        wait $PID 2>/dev/null
        if [ $indice -gt 0 ]; then
            indice=$((indice - 1))
        else
            echo "Te encuentras en la primer cancion!"
            indice=0
        fi
    else
        wait $PID
        indice=$((indice + 1))
    fi
done

echo "Playlist finalizada. Hasta luego!"

\end{lstlisting}
\pagebreak
\subsection{Script: salir}

El script de salir está diseñado para mostrar un mensaje visual en el terminal y luego finalizar un proceso asociado a una variable de entorno. En la primera parte, el script imprime una serie de cadenas de texto que forman una representación artística (un banner) utilizando caracteres ASCII. Tras mostrar el banner, el script espera unos segundos,y luego verifica si existe un valor en la variable de entorno. Si existe un proceso con ese ID, el script lo termina usando el comando kill-9.
Este script se ejecuta cuando el usuario ingresa el comando \texttt{salir} en la terminal.

\subsubsection{Captura de pantalla de la ejecución}

\begin{figure}[h!]
\centering
\includegraphics[width=0.7\textwidth]{salir.png}
\caption{Ejecución del script de salir}
\end{figure}

\subsubsection{Código fuente del script}

\begin{lstlisting}[language=bash]
#!/bin/bash

echo ""
echo "CERRANDO TERMINAL, BYE"
echo ""
sleep 2

if [[ -n "PROYECTO_PID" ]]; then
    kill -9 $PROYECTO_PID
else 
    echo "ERROR"
fi 

\end{lstlisting}
\pagebreak

\section{Conclusiones}

\subsection{Conclusión de Frida}

Este proyecto ha sido una experiencia significativa en muchos aspectos. En primer lugar, aprendí a trabajar con Bash y a entender mejor el funcionamiento interno de una terminal de Linux. Sin embargo, también me permitió desarrollar mis habilidades de trabajo en equipo, ya que tanto mi compañero como yo tuvimos que aplicar conceptos clave como organización, priorización, comunicación y responsabilidad. Estos elementos fueron esenciales para poder coordinar las tareas de manera efectiva, asegurándonos de que todas las partes del proyecto se completaran de acuerdo a lo planificado y dentro de los plazos establecidos.

Aunque nunca antes había tenido contacto con Linux, este proyecto me dio la oportunidad de aprender y profundizar en el uso de un sistema operativo que antes me resultaba desconocido. Desde la instalación y configuración del entorno hasta la implementación de comandos personalizados, cada paso me permitió mejorar mis habilidades técnicas. Me enfrenté a varios retos, como implementar el sistema de inicio de sesión y estructurar adecuadamente los scripts, pero despues de varios intentos logré resolverlos. Considero que este proyecto fortaleció mis habilidades en programación de scripts y en documentación técnica mediante LaTeX. Pero más allá de los aspectos técnicos, lo que más me impactó fue el desarrollo de mis habilidades prácticas. La constante interacción con el sistema me permitió practicar y entender conceptos clave lo cual me ha proporcionado una base sólida para proyectos futuros.

En resumen, este proyecto no solo me ha permitido adquirir nuevos conocimientos en Linux y programación, sino que también me ha ayudado a fortalecer mi capacidad de trabajar en equipo, coordinar tareas, y afrontar desafíos con responsabilidad. Estoy segura de que esta experiencia será valiosa para mi desarrollo profesional y académico, y me ha motivado a seguir aprendiendo y enfrentando nuevos retos.

\subsection{Conclusión de Cristopher}

El desarrollo de habilidades con sistemas operativos basados en Unix es muy importante a la
hora de trabajar en cualquier ámbito. Funciona muy bien como un sistema como alternativa al
típico Windows y algunas distribuciones pueden ofrecer muchas mejoras frente al sistema que
siempre se utiliza.
No solo es importante tener instalado el sistema en si mismo. Si bien el manejo es muy similar
al de cualquier otro sistema Windows, si se aprende a usar herramientas mas especificas
dentro del sistema como el manejo de la terminal, de los servicios, y de los scripts en bash,
pueden resultar una herramienta muchísimo mas practica y eficaz al momento de trabajar.
Es bien sabido que en Windows, a la hora de tener que instalar nuevas herramientas como
compiladores o programas, se deben de seguir cierta serie de pasos específicos que en ciertas
ocasiones pueden resultar confusas de seguir, y otra característica importante de Linux es que
este tipo de problemas se eliminan por completo a la hora de utilizar un gestor de paquetes.
Desde luego que usar un sistema Linux, y comprender la mayoría de las funciones básicas son
habilidades que cualquier usuario debería de aprender para que su productividad mejore
significativamente.

Finalmente, cabe destacar que hay ciertas situaciones donde usar un sistema diferente como
Windows podría resultar en un resultado mejor, algo muy típico y que siempre se discute es a
la hora de usar la suite de aplicaciones de Adobe, sin embargo, se trata de casos particulares.
La cantidad de información que se puede encontrar en internet acerca del uso de este tipo de
sistemas es bastante extensa, con una gran cantidad de foros, artículos y enciclopedias que
cualquier usuario, tanto novato como experimentado puede consultar, por lo tanto, los
conocimientos adquiridos servirán mucho en el futuro para diversas actividades, desde el
desarrollo de aplicaciones y programación, hasta cosas mas sencillas como la edición de
documentos, uso de aplicaciones de ofimática o navegar en internet.
\end{document}
